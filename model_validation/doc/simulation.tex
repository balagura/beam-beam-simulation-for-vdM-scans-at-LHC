\documentclass[paper=a4, fontsize=11pt]{scrartcl} % A4 paper and 11pt font size

\usepackage[T1]{fontenc} % Use 8-bit encoding that has 256 glyphs
\usepackage[english]{babel} % English language/hyphenation
\usepackage{amsmath,amsfonts,amsthm} % Math packages

\usepackage{sectsty} % Allows customizing section commands
\allsectionsfont{\centering \normalfont\scshape} % Make all sections centered, the default font and small caps
\usepackage{graphicx} % Required to insert images

\setlength{\headheight}{13.6pt} % Customize the height of the header

\numberwithin{equation}{section} % Number equations within sections (i.e. 1.1, 1.2, 2.1, 2.2 instead of 1, 2, 3, 4)
\numberwithin{figure}{section} % Number figures within sections (i.e. 1.1, 1.2, 2.1, 2.2 instead of 1, 2, 3, 4)
\numberwithin{table}{section} % Number tables within sections (i.e. 1.1, 1.2, 2.1, 2.2 instead of 1, 2, 3, 4)

%\setlength\parindent{0pt} % Removes all indentation from paragraphs - comment this line for an assignment with lots of text

\newcommand{\horrule}[1]{\rule{\linewidth}{#1}} % Create horizontal rule command with 1 argument of height

\title{
\normalfont \normalsize
%\textsc{} \\ [25pt] % Your university, school and/or department name(s)
%\horrule{0.5pt} \\[0.4cm] % Thin top horizontal rule
\huge Beam-beam simulation for van der Meer scans  at LHC\\ % The assignment title
\horrule{2pt} \\[0.5cm] % Thick bottom horizontal rule
}

\author{Vladislav Balagura} % Your name

\date{\normalsize\today} % Today's date or a custom date

\begin{document}

\maketitle % Print the title

\section{Simulation method}


At any LHC IP:
\[ u = \sqrt{\epsilon \beta^*} \cos(\phi), \]
\[ u' = du/ds = -\sqrt{\epsilon /\beta^*} \sin(\phi), \]
since $\beta = \beta^*$ is a minimum of the beta-function, so $d \beta/ds = 0$.
Here, $u$ denotes either $x$ or $y$ transverse coordinate, 
$s$ is the longitudinal coordinate, $\epsilon$ - the emittance.
It is convenient to define a complex variable
\[ z = u - i\beta^* u' = \sqrt{\epsilon \beta^*} e^{i \phi}. \]
One turn of the particle in bunch 1 is described in the simulation:
\[ z_{n+1} = (z_n - i \beta^* \Delta u') e^{2\pi i Q}. \]
This is done separately in $x$ and in $y$. The angular kick $\Delta u'$
introduces the coupling between them, since it depends on both $x$ and $y$ as
\[ \Delta u'_{x,y} =
-\frac{2 Z_1 Z_2 \alpha \hbar N_2 \beta^*}{\beta_0 p} \frac{\vec{R}}{R^2} (1 -
e^{-R^2/2\sigma_2^2}). \] Here, $p = m\beta_{rel}\gamma$ is the momentum of
the particle, $\beta_0 =\frac{2\beta_{rel}}{1+\beta_{rel}^2} \approx 1$ is the
relativistic velocity of the second bunch in the rest frame of the
first. $\vec{R}$ is the bunch separation vector pointing from the center of
the first bunch to the center of the second, $N_2$ is the number of particles
in the second bunch.

It is assumed that the second bunch is unperturbed. This is justified by the
argument that its deformations are of the order O(1 $\mu m$) while the
characteristic bunch sizes and the distance between them is O(100 $\mu m$). For
simplicity, the round bunches are considered at the moment (as in the old
MAD-X dynamic-beta simulation), but Basetti-Erskine formula can be implemented
in the future.
 
The main problem of such simulation is to obtain a good accuracy of 0.1\%
or better with a reasonable number of sampling points and a minimal CPU time.

Particles of the first bunch are traced individually. In the current model,
typically, 10 000 particles each making 10 000 turns are sufficient.
Beam-beam interaction is switched on not immediately but after 1000
turns. This allows to calculate numerically the undisturbed overlap integral,
compare it with the exact analytic formula and estimate the bias of the
numerical integration. The final correction is then calculated as the ratio of
the numerical integration with and without the beam-beam interaction. In this
ratio the potential bias cancels at least partially, so this helps to improve
precision. But in the end, the bias was found negligible, however.

The traced particles of the 1st bunch are distributed in the following
way. The first bunch Gaussian is sampled in two-dimensional X-Y grid covering
$\pm 5\sigma \times (\pm 5\sigma)$ rectangle with $\sqrt{10 000} = 100$ points
along each side. Then, all points farther than $5\sigma$ from the center of
the first or the second bunch center are removed. The remaining grid of
$r_x^i$, $r_y^i$ points is two-dimensional, while one needs to cover
4-dimensional phase space $(x,x',y,y')$. To do that, every point is rotated
with the random phases $\phi_x^i$, $\phi_y^i$ uniformly distributed in $[0,\
2\pi]$:
\[ x_i - i\beta^* x'_i = r_x^i e^{i \phi_x^i}, \]
\[ y_i - i\beta^* y'_i = r_y^i e^{i \phi_y^i}. \]

During first 1000 accelerator turns without beam-beam the $i$-th point "fills"
the circle in $(x, -\beta x')$ complex plane with radius $r_x^i$ and similarly
in $y$. This gives the necessary sampling of every circle, essentially, the
number of sampling points is 10 000 particles $\times$ 1000 turns. This allows
to achieve sufficient density of points in four dimensions for precise
numerical Monte Carlo estimation of the luminosity integral.  Note, this
integral is taken in $x,y$ two-dimensional space which is a projection of
4-dimensional space. However, it requires sampling of the full 4-dimensional
space.

After beam-beam is switched on, there is a "stabilization" period. Typically,
it is rather short, at least less than 2 000 turns.  The remaining 3 000 -- 10
000 turns are then also used for averaging. If the beam-beam force is not too
large, as in typical vdM scans, particles still move approximately around the
circles, so averaging over these turns effectively increases the sampling.

It is checked that the overlap integrals averaged over groups of 100 turns
after the stabilization period give the same results modulo statistical
fluctuations. The final result is taken as an average over the full 3 000 --
10 000 turns period.

In calculating the overlap integrals the central points should be taken with
larger weight than the peripheral.  The weight is determined from the
distribution of radii in two-dimensional Gaussian eg. in $(x,x')$: \[ w_x^i
\propto r_x / \sigma^2 \exp(-r_x^2/2\sigma^2) dr_x. \] The weights in $x$ and
in $y$ are multiplied, $w_i = w_x^i w_y^i$, and normalized, $\sum_i w_i = 1$.

Finally, the overlap integral is calculated as a sum of particle weights times
the 2nd bunch profile density at the point of the particle, ie. as $\sum_i w_i
\rho_2(x,y)$.



\end{document}
